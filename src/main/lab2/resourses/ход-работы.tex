\subsection{Аналитическое решение}\label{subsec:аналитическое-решение}
Аналитически найдём минимум функции {$f(x,y) = 4x^2 + 6xy + 4y^2 + 8x + 10y + 1$}.
\newline
C помощью WolframAlpha получаем точку минимума \{$[?$\}.

\subsection{Общая схема поиска}\label{subsec:общая-схема-поиска}
Будем строить последовательность точек {$x^k$}, сходящихся к точке минимума $x$. \newline
Для этого на каждом шаге будем находить \newline
{$x^{k+1}=x^k+\alpha^{k}*p^k$}, где {$p^k$} - вектор направления убывания, {$\alpha^k$} > 0, \newline
{$f(x^k+\alpha^k*p^k) < f(x^k)$}.\newline
Критерием окончания поиска считаем: ||\nabla {f(x^k)}||< {\varepsilon}.
\newline

\subsection{Метод градиентного спуска}\label{subsec:метод-градиентного-спуска2}
Метод градиентного спуска заключается в том, что на каждом шаге мы пытаемся пойти по направлению наискорейшего убывания функции(антиградиенту).
Идем с шагом $\lambda$, уменьшаем его в два раза, если не попали в меньшую точку. \newline
Рассмотрим метод работы градиентного спуска на примере {$f(x,y) = 4x^2 + 6xy + 4у^2 + 8x + 10у + 1$}.
Начальное приближение: точка {$x = (1, -1)$}. \newline
Начальное значение $\lambda$ = 0.01. \newline
Число итераций: $?$. \newline
Конечный модуль градиента: $?$.\newline
Результат: $Xmin = (?)$, $Fmin = ?$.
\newline
\newline
Таблица, отражающая скорость сходимости, и график сходимости:

\subsubsection{Вывод}
В градиентном методе с дроблением шага параметр $\lambda$ выбирается изначально и в последующих итерациях делится на два, что позволяет быстро пересчитывать $\lambda$ на каждом шаге.
Для более точного результата начальное значение $\lambda$ должно равняться {$2/(l + L)$}, где $l$ и $L$  - наибольшое и наименьшее собственное число матрицы $A$.
Однако, нахождение собственных значений произвольной матрицы - задача нетривиальная и ресурсозатратная, поэтому мы фиксируем $\lambda = 0.01$.
\newline
\newline
Как видно из таблицы, вблизи $Xmin$ скорость сходимости заметно убывает.
Это связано с тем, что модуль градиента становится малой величиной, засчет чего график сходимости имеет зигзагообразный характер.
Одним из вариантов решения этой проблемы является нормализация градиента, однако это требует дополнительных вычислений.
\newline
\newline
Обычно вычисление градиента является более ресурсозатратной операцией, чем вычисление значения функции, поэтому метод градиентного спуска не является оптимальным.

\newpage